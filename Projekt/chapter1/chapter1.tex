\chapter{Linear Programming}
In this chapter we introduce the concept of Linear programming. Most proofs will be omitted but proofs and more in depth explanations can be found in ~\cite{vanderbei2015linear}
\section{The structure of a linear program}
In a linear program we want to maximise or minimize a given linear function $z:\R^n \rightarrow \R$ subject to a number of linear inequalities or equalities $g_i:\R^n \rightarrow \R$ with $g_i(x_1,...,x_n)\geq b_i$,$g_i(x_1,...,x_n)\leq b_i$ or $g_i(x_1,...,x_n)= b_i$. The function, $z$, that we want to optimise is called the \textbf{Objective} and the set of inequalities are called the \textbf{Constraints}. A general linear problem is written:
\begin{align}\label{lpform}
\begin{array}{ll@{}ll}
\text{min/max} &z(x_1,...,x_n)&\\
\text{s.t.} &g_1(x_1,...,x_n) \textbf{ ordRel } b_1,&\\
&\vdots&\\
&g_m(x_1,...,x_n) \textbf{ ordRel } b_m&\\
&\text{where each \textbf{ordRel} can be }\leq,\geq \text{ or } =&
\end{array}
\end{align}

\begin{example}\label{lpex}
A maths student wants to save money on his diet while still remaining healthy. To stay healthy his diet must contain at least $b_1=6$ units of protein, $b_2=15$ units of carbs, $b_3=5$ units of fat and $b_4=7$ units of vitamins.\\
He consider buying 3 food products with different nutritional values and prices: 
\begin{enumerate}
\item $x_1$ is a take away meal costing 5 and containing 3 units of protein, 3 units of carbs, 2 units of fat and 1 unit of vitamins.
\item $x_2$ is a vegetable costing 1 and containing 1 unit of protein, 2 units of carbs, 0 units of fat and 4 units of vitamins.
\item $x_3$ is a type of bread costing 2 and containing $\frac{1}{2}$ unit of protein, 4 units of carbs, 1 unit of fat and 0 units of vitamins.
\end{enumerate}
He then define an optimization problem minimizing the cost of food subject to getting the right nutrition
\begin{align}
\begin{array}{ll@{}ll}
\text{min} &5x_1+x_2+2x_3&\\
\text{s.t.} &3x_1+x_2+\frac{1}{2}x_3 \geq 6,&\\
&3x_1+2x_2+4x_3 \geq 15,&\\
&2x_1+x_3 \geq 5,&\\
&x_1+4x_2 \geq 7,&\\
\end{array}
\end{align}
\end{example}
\section{feasible and optimal solutions}
\subsection{feasibility}
Given a Linear problem on form \ref{lpform} any point of $\R^n$ such that all $m$ constraints are true is called a feasible solution. The set of all these points is called the feasible set. In the case \Cref{lpex} the feasible set is the set of all combinations of amounts of the different foods such that the nutritional requirements are met. If a problem has no feasible solutions, that problem is said to be infeasible. A feasibility problem is a special case in linear programming where our object function is constant and thus if any feasible solution exists, that solution is optimal.
\subsection{optimal solutions}
A feasible solution $\textbf{x}_0 \in \R^n$ is said to be optimal if $z(\textbf{x}_0) \geq z(\textbf{x}_0)$ (when maximizing) or $z(\textbf{x}_0) \leq z(\textbf{x}_0)$ (when minimizing) for all feasible solutions $\textbf{x} \in \R^n$.\\
In some instances when feasible solutions exist, but no maximal (or minimal) solution exists the problem is said to be \textit{unbounded}. In that case one or more variable in the objective can approach $\infty$ or $-\infty$ in a solution, all while that solution remains feasible and the objective value diverges.\\\\ 
\todo{convexity and simplex part.}
\section{Convexity}
\begin{definition}\label{convex}
A set $X \in \R^n$ is said to be convex if for any two points $a, b \in X$ the straight line segment 
%$\{x|x=(1-t)a+tb,t\in [0,1]\}$
connecting $a$ and $b$ is entirely within $X$.
\end{definition}
\begin{theorem}
A feasible set of a linear program is convex
\begin{proof}
The feasible set of an LP in $\R^n$ is exactly the intersection of all the halfspaces given by each constraint function. \\
Since such halfspaces are convex and the intersection of convex sets is also convex, the entire feasible set is convex.
\end{proof}
\end{theorem}
\begin{lemma}
Any solution between two feasible solutions is feasible.
\begin{proof}
Since the feasible region of an LP is convex, any point between two feasible points is within the feasible region.
\end{proof}
\end{lemma}
\subsection{The Fundamental Theorem of Linear Programming}
~\cite[Theorem 3.4]{vanderbei2015linear}
\begin{theorem}\label{extreme point}
If a set $\mathcal{S}$ of optimal solutions to a given LP exists, then some $s\in \mathcal{S}$ such that $s$ is in an extreme point in the feasible set.
\begin{proof}
Let $A$ be the feasible region to a given Linear problem and let $\textbf{x}_0=\spvec{x_1;\vdots;x_n}$ be an optimal solution with value $z(\textbf{x}_0=\zeta$.
Suppose $\textbf{x}_0$ is not in an extreme point of $A$. Then $\textbf{x}_0$ is either in the open set $A'=A-\bar{A}$ or on some point of $\bar{A}$ that is not an extreme point. In the first case at least one of the coordinates $x_1,\cdots, x_n$ can be increased or decreased until the solution reaches a point $\textbf{x}_1$ on $\bar{A}$ with $z(\textbf{x}_1)< \text{ or } > \zeta$. Thus that cannot be the case.\\
In the second case there are two options;
\begin{enumerate}
\item The hyperplane defined by $z(\textbf{x}) = \zeta$ intersects one or more of the constraint functions in $\textbf{x}_0$ but is not parallel to any of them.
\item The hyperplane defined by $z(\textbf{x}) = \zeta$ is equal to the entire hyperplane defined by a constraint function or all points of the intersection of several constraint functions are contained within the hyperplane.
\end{enumerate}
In the first case we have a contradiction since we can increase or decrease the values $x_1,\cdots, x_n$ while still remaining in the feasible region and thus find a vector $\textbf{x}_1$ with $z(\textbf{x}_1)< \text{ or } > \zeta$. In the second case the entire subset of the hyperplane contained in the feasible region is optimal and thus the extreme points of that set are also optimal solutions.
\end{proof}
\end{theorem}
\section{The geometric/graphical intuition}
From a geometric perspective the feasible region of an LP can be seen as a convex polyhedron encapsulated by the hyperplanes defined by each constraint function And the objective is a hyperplane that can be "pushed" orthogonally to either maximize or minimize a point of the plane such that it is contained in the feasible polyhedron.\\ 
In case of \Cref{lpex}, with three real variables, $x_1,x_2,x_3$ the polyhedron will be a polyhedron in $\R^3$ and the objective plane will be the plane defined by the linear equation $5x_1+x_2+2x_3 = \zeta $ where $\zeta $ is the value we want to minimize. \todo{TikZ of \Cref{lpex}}
\section{Matrix representation of an LP and Slack variables}
Since the objective function and the constraints of a linear program are all linear functions we can also write a linear program using matrix-vector products. \\
The object can be written as the product of the vector of variables to be determined 
$\begin{bmatrix}x_{1} \\           x_{2} \\
\vdots \\
x_{n}
\end{bmatrix}=\textbf{x}$, and the transposed vector of coefficients $[c_1,c_2,...,c_n]=\textbf{c}^T$ of each $x_i$ in the objective.\\
Likewise the constraints can be written as the matrix-vector products of $n\times m$ constraint matrices,$A$ where $m$ is the number of a constraints with a certain order relation, and $\textbf{x}$ with some order relation ($\leq,\geq or =)$ to a $1\times m$ vector $\textbf{b}$.\\
Rewriting any $\leq$ constraint to $\geq$ in case of minimization or any $\geq$ to $\leq$ in case of maximization (the equality constraints can be written with two inequalities) we can write the linear program on it's \textit{canonical form}:
\begin{align}\label{canonicalMin}
\begin{array}{ll@{}ll}
\text{min} &\textbf{c}^T\textbf{x}&\\
\text{s.t.} &A\textbf{x} \geq \textbf{b},&\\
&\textbf{x} \geq 0,&\\
\end{array}
\end{align}
\begin{align}\label{canonicalMax}
\begin{array}{ll@{}ll}
\text{max} &\textbf{c}^T\textbf{x}&\\
\text{s.t.} &A\textbf{x} \leq \textbf{b},&\\
&\textbf{x} \geq 0,&\\
\end{array}
\end{align}
in case of \Cref{lpex} we write it on it's canonical form:
\begin{align}
\begin{array}{ll@{}ll}
\text{min} &[5,1,2]\begin{bmatrix}x_{1} \\x_{2} \\x_{3}\end{bmatrix}&\\
\text{s.t.} &\begin{bmatrix}3&2&1 \\12&2&4 \\7&0&2\\3&5&2\end{bmatrix}\begin{bmatrix}x_{1} \\x_{2} \\x_{3}\end{bmatrix} \geq \begin{bmatrix}5 \\10 \\5\\10\end{bmatrix},&\\
&\textbf{x} \geq 0,&\\
\end{array}
\end{align}
Much like how we made an LP on canonical form by manipulating the inequalities, we can also write any LP on it's \textit{standard form} 
\begin{align}
\begin{array}{ll@{}ll}
\text{max/min} &\textbf{c}^T\textbf{x}&\\
\text{s.t.} &A\textbf{x} = \textbf{b},&\\
&\textbf{x} \geq 0,&\\
\end{array}
\end{align}
by introducing a $1\times m$ vector, $\textbf{s}$, with one slack variable for each constraint and thus creating a new LP on standard with the same solutions as the old one.
\begin{align}
\begin{array}{ll@{}ll}
\text{max} &\textbf{c}^T\textbf{x}&\\
\text{s.t.} &A\textbf{x}+\textbf{s} =\textbf{b},&\\
&\textbf{x,s} \geq 0,&\\
\end{array}
\end{align}
These forms and the concept of slack variables will come in handy later (I hope)
\section{Duality}
Another important result in Linear programming is the concept of duality.
\begin{definition}
Given a maximization linear program on canonical form \ref{canonicalMax}
We define it's dual problem as the minimization problem:
\begin{align}\label{canonicalMaxDual}
\begin{array}{ll@{}ll}
\text{min} &\textbf{b}^T\textbf{y}&\\
\text{s.t.} &A^T\textbf{y} \geq \textbf{c},&\\
&\textbf{y} \geq 0,&\\
\end{array}
\end{align}
\end{definition}
\begin{theorem}
An optimal solution to a linear program on it's primal form is also optimal in the dual form
\begin{proof}
\todo{proof}
\end{proof}
\end{theorem}
\subsection{Weak and strong duality}
\section{Efficiency of solving an LP}
\todo{write this part}
This might be relevant since linear programming is in P and integer programming is NP-hard.
\subsection{The simplex method}
A popular algorithm for finding the optimal solution of LPs, taking advantage of the convexity of the feasible region and \Cref{extreme point} is the \textbf{Simplex method}. ~\cite{vanderbei2015linear}, where Theorem 3.4 also gives a proof of \ref{extreme point} using the simplex method, explains this further.
\\\\ In the next chapter we will introduce Integer programming which is NP-hard.
\chapter{Integer Programming and Graphs}
In this chapter we introduce the concept of Integer programming. We will approach the subject with examples and results from graph theory, but apart from fundamental definitions we will omit most of the theory from this field of mathematics, but a more in depth explanation can be found in ~\cite{wolsey1998integer}. 
\section{Some Graph notation}
\begin{definition}\label{cromatic number}
A \textbf{graph} $G$ is an ordered set $(V,E)$ where $V$ is a set of vertices and $E$ is a set of edges and each edge in $E$ is a subset of $V$ containing two vertices.
\end{definition}
\begin{definition}
let $G=(V,E)$ be a graph. Two vertices $v,u\in V$ are said to be \textbf{adjacent} if the edge $(v,u)$ (or $(u,v)$) is in $E$.
\end{definition}
\begin{definition}
let $G=(V,E)$ be a graph. A subset $S$ of $V$ form an \textbf{independent set} if no two vertices in $S$ are adjacent in $G$
\end{definition}
\begin{definition}
let $G=(V,E)$ be a graph. A subset $S$ of $V$ form a \textbf{clique} if all pairs of two vertices in $S$ are adjacent in $G$
\end{definition}
\begin{definition}
let $G=(V,E)$ be a graph. A \textbf{maximal independent set} or a \textbf{maximal clique} is an independent set or a clique respectively where if any other vertex $v \in V$ is added to the set, it will lose the property of being an independent set or clique respectively 
\end{definition}
\begin{definition}
A \textbf{maximum independent set} or a \textbf{maximum clique} in a graph $G$ is an independent set or a clique respectively where no other independent set or clique in $G$ respectively have more vertices.
\end{definition}
\begin{definition}
the \textbf{clique number} $\omega(G)$ of a graph $G$ is the number of vertices in a maximal clique in $G$.
\end{definition}

\section{Mixed Integer Programming - MIP}
A Mixed integer problem (MIP) is a way of solving optimisation problems with integer values. The Structure of an integer problem is the same as a linear problem, but with the extra constraint that all or some of the variables are in $\Z$. This added constraint might seem insignificant, but it introduces crucial differences to problems with real variables. One significant difference is that the feasible space of an MIP consists of discrete sets and thus it is not convex.
\todo{tix showing that it is not convex?}
\subsection{Integer and Binary Constraints}
An important method in Integer programming is \textbf{Binary programming}. Here variables take values either $1$ or $0$ these binary variables are very useful since they can easily be used to say "Is variable $x_i$ part of the solution? then $x_i = 1$ otherwise $x_i = 0$. Binary constraints will be used extensively once we start colouring Graphs, but another example of their usefulness is in determining maximal cliques and independent sets of graphs:
\begin{example}\label{clique}
Maximal clique and maximal independent set problem:\\
Let $G=(V,E)$ be a graph and let $H=(V,E')$ be it's complement graph. Then
\begin{align}
\begin{array}{ll@{}ll}
\text{max} &\sum_{v\in V} x_v&\\
\text{s.t.} &x_u + x_v \leq 1,& \forall (u,v) \in E\\
&x_v\in\{0,1\},& \forall v \in V\\
\end{array}
\end{align}
calculates the maximal independent set and
\begin{align}
\begin{array}{ll@{}ll}
\text{max} &\sum_{v\in V} x_v&\\
\text{s.t.} &x_u + x_v \leq 1,& \forall (u,v) \in E'\\
&x_v\in\{0,1\},& \forall v \in V\\
\end{array}
\end{align}
Calculates the maximal clique
\todo{tikz and proof}
\end{example}
\section{Relaxation}
\todo{how is it usefull? what does the relaxation of \ref{clique} mean)}
\section{Integer Programming is NP-hard}
\todo{\ref{clique} is an NP-hard problem}